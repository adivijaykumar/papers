
\documentclass[a4paper,11pt]{article}

\usepackage{physics}
\usepackage{amsmath}
\usepackage{amssymb}
\usepackage{amsmath}
\usepackage{amsthm, mathtools}
%\usepackage{hyperref}
\usepackage{color}
\usepackage{jheppub}
\usepackage[T1]{fontenc} % if needed

% My Documents
\newcommand{\be}{\begin{equation}}
\newcommand{\ee}{\end{equation}}
\newcommand{\bes}{\begin{equation*}}
\newcommand{\ees}{\end{equation*}}
\newcommand{\bea}{\begin{flalign*}}
\newcommand{\eea}{\end{flalign*}}


%\linespread{1.0}
%\setlength{\parindent}{0em}
%\setlength{\parskip}{0.8em}

\title{\textbf{Notes for BH Mimickers Project}}
\author{Aditya Vijaykumar}
\affiliation{International Centre for Theoretical Sciences, Bengaluru, India.}
\emailAdd{aditya.vijaykumar@icts.res.in}

\begin{document}
\maketitle

The basic aim is to ascertain whether the GWs detected by LIGO were really due to the coalescence of binary black holes. Put another way - we want to assume some simple models of these alternative objects, and constrain what the parameters of these models might be.

\section{Reviews}

\subsection{Notes on arXiv:1804.08026 - Constraining black hole mimickers with gravitational wave observations}

\begin{itemize}
	\item A commonly considered alternative to Black Holes are Boson Stars (hereforth abbreviated as BS; no pun intended). BS are solutions of Einstein equations with the matter term containing a complex scalar field coupled to the metric. One can write the Einstein-Hilbert action describing the BS as follows,
	\begin{equation*}
	S = \int \dd^4{x} \sqrt{-g}\qty[R - \dfrac{1}{2} \qty(g^{ab} \partial_a \phi \partial_b \phi   + V(\abs{\phi}^2))]
	\end{equation*}
	The scalar field gives the BS energy to gravitate. It is not exactly very straightforward why the BS should be stable, and what should balance the gravitation. The bottom line, though, is that one can think of scenarios where BS might be stable against gravitational collapse. \textit{Living Reviews in Relativity} has what seems like a good summary of BS. \footnote{\text{https://link.springer.com/article/10.1007\%2Fs41114-017-0007-y}} \textit{Gravastars} are another proposed model for such compact objects.
	
	\item One is not sure about what electromagnetic signatures such models can have, but one can have hope of detecting (or not detecting) these objects through Gravitational Waves. Such non-BH models will have tidal deformation due to the companion object, as well as deformations due to their own spin. These deformations would change the gravity around these objects, and hence the gravitational waves that are emitted from them. We can hope of detecting these deviations from Black Holes in the inspiral waveform.
	
	We choose the inspiral because we think we understand the PN expansion very well. We essentially cut off our analysis before PN starts losing significance. One can't deal with the merger phase very well as full numerical simulations of these models are still at a very nascent stage. \textcolor{red}{But can one deal with such objects in perturbation theory and obtain ringdown modes for such models?} \footnote{https://journals.aps.org/prd/pdf/10.1103/PhysRevD.50.6235}
	
	\item As opposed to the Fisher Matrix analysis used in other papers, this paper makes use of a full Bayesian method using the best GW waveforms. \textcolor{red}{What are the problems with the Fisher Matrix analysis and why is this method better?}
	
	\item The objects considered in this paper are nonspinning, perfect fluid stars described by a polytropic equation of state. We have,
	\begin{equation*}
	p = K \rho^{1 + 1/n} \qq{and} \rho = \epsilon - n p
	\end{equation*}
	where $ p $ is the pressure, $ \rho $ is the rest mass density, $ n $ is the polytropic index, $ \epsilon $ is the energy density. For this analysis, we require the mass-radius relationship and the mass-tidal deformability relationship - these are the quantities that determine the contact frequency for such systems. \textcolor{red}{Read up why this is so.} The waveforms can be modelled using the quadrupolar tidal deformability, but ascertaining the contact frequency requires higher multipolar moments. The stellar structure is computed by solving TOV and using existing expressions to find tidal deformabilities.
	
	\item At the start of the inspiral, we can approximate the system as two-body point particle motion, but as we proceed towards the late inspiral, the tidal effects will become important and the PN expansion will be rendered untrustworthy. To account for the tidal effects, we add tidal corrections to the IMRPhenomD waveforms \textcolor{red}{How do I access the IMRPhenomD waveforms? Also, what is the idea behind phenomenological waveforms, and what are its shortcomings?} at some order in the expansion. \textcolor{red}{How do we figure out the order? Naive guess would be that doing a PN-like expansion for these tidally deformed systems would give us valid corrections only at some specific order, and hence we take that.} Even after this, due to the approximations that we have made, we cannot expect the waveform to be valid when the objects come into contact. Hence, we cut off our analysis at that point. We take this point to be the frequency at the first instance when one of the stars has $ \lambda = 0.2 $ \textcolor{red}{Why this number and not any number? Is there any calculation which shows that this is a reasonable approximation to make?}, and call it the \textit{contact frequency}.
	
	\item The rest of the paper is mostly technicalities of the data analysis. Plots are drawn of the combined posteriors $ \Lambda $ and $ n $. \textcolor{red}{The nitty gritties of the data analysis is something that I don't understand well, and something I need to figure out soon.} And a certain class of models with $ K, n $ are ruled out. The interesting thing is that the authors claim that \textit{boson stars can approximately thought to be like polytropic stars} by looking at the $ \Lambda $ vs $ n $ curves, and hence we can constrain the potentials of these boson stars. \textcolor{red}{But really, is that a fair enough argument? Isn't there a way to be more precise?} We then assume of $ \phi^4 $ potential and translate the constraints obtained from polytropic stars to constraints on $ m_B, \lambda_B $ in the $ \phi^4 $ potential. 
	
	\item The authors say they want to extend the current work to include gravastar models as also to the spinning case. \textcolor{red}{What exactly are the complications in doing the same?}
\end{itemize}
\end{document}