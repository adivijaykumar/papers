
\documentclass[a4paper,11pt]{article}

\usepackage{physics}
\usepackage{amsmath}
\usepackage{amssymb}
\usepackage{amsmath}
\usepackage{amsthm, mathtools}
%\usepackage{hyperref}
\usepackage{color}
\usepackage{jheppub}
\usepackage[T1]{fontenc} % if needed

%\linespread{1.0}
%\setlength{\parindent}{0em}
%\setlength{\parskip}{0.8em}

\title{\textbf{Eigenstate Thermalization Hypothesis}}
\author{Aditya Vijaykumar}
\affiliation{International Centre for Theoretical Sciences, Bengaluru, India.}
\emailAdd{aditya.vijaykumar@icts.res.in}

\begin{document}
\maketitle
\section{Motivation}
Lets look at unitary thermalization. One has a closed quantum system in a \textit{big} box, \textit{eg}. box of gas. Let the state of the system be described by 
$$\ket{\psi(t)} = \exp(-i H t/\hbar ) \ket{\psi_0}$$ Quantum systems in general would look thermal at large times $t$. ETH is an explanation of this fact.

What does \textit{look thermal} mean? The state we are considering is a pure state, but if one considers measuring an observable at large times, the claim is that it would look the same as it would when measured in the Gibbs ensemble.

Lets say that $\ket{\psi_0} = \sum_{k} c_k \ket{k}$ and $H\ket{k} = E_k \ket{k}$. An observable $A$ is to be measured. Let $A(t)$ be the expectation value of the observable at time $t$.
$$A(t)  = \sum_{k,l}c_k c_l^* \bra{l}A \ket{k}$$ \textcolor{red}{Some error here}. We expect that $A(t)$ will evolve to an equilibrium value and after waiting for some time and according to the Poincare recurrence theorem, $A(t)$ will come back to its initial value. To calculate the equilibrium value, we just take the time average of $A(t)$,
$$\bar{A}=\lim\limits_{t \rightarrow \infty} \frac{1}{T}\int_{0}^{T} A(t) dt = \sum_{k} A_{kk}\abs{c_k}^2$$
\textcolor{red}{Do the integration - oscillatory terms cancel}
Suppose $\ket{\psi_0}$ is a superposition of states with not very different energies, and make the following hypothesis,\\
\textbf{Hypothesis 1} - For $E_k \in [E-\Delta, E]$, $A_{kk} = A^{(E,\Delta)} + \delta A_{kk}$, where $\delta A_{kk}$ is small.

Substituting into $\bar{A}$, we get $\bar{A} = A^{(E,\Delta)}$. One can also calculate a corresponding microcanonical average from statistical mechanics, $\expval{A} = \frac{1}{N}\sum_{k} A_{kk}= A^{(E,\Delta)}$. Hence the time average is equivalent to the thermal average if \textbf{Hypothesis 1} is true. This means that the system will behave like what one would expect from statistical physics. 

But the equilibrium value is not all that we need. We would also like to make a statement about the fluctuations at large times, and we would expect them to be small. This would be given by the quantity \textcolor{red}{Do the calculation}$$(A(t) - \bar{A})^2 = \sum_{k\ne l; E_k, E_l \in [E-\Delta,E]} \abs{c_k}^2 \abs{c_k}^2 \abs{A_{kl}}^2 \ll \bar{A}^2$$
\end{document}